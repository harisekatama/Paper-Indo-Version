% Ubah judul dan label berikut sesuai dengan yang diinginkan.
\section{Hasil Pengujian}
\label{sec:hasil pengujian}

% Ubah paragraf-paragraf pada bagian ini sesuai dengan yang diinginkan.
Pada bab ini akan dipaparkan mengenai beberapa skenario pengujian sesuai dengan telah dijelaskan pada metodologi. Skenario pengujian ini dilakukan guna untuk mengetahui waktu \emph{delay} yang dibutuhkan untuk mentransmisikan data dari laptop menuju ESP32. Skenario yang nantinya akan diterapkan pada pengujian meliputi beberapa poin sebagai berikut:

\begin{enumerate}
  \item Pengujian waktu \emph{delay} pengiriman data String melalui Bluetooth
  \item Pengujian waktu \emph{delay} pengiriman data JSON melalui Bluetooth
  \item Pengujian waktu \emph{delay} pengiriman data String melalui \emph{Access Point} WiFi
  \item Pengujian waktu \emph{delay} pengiriman data JSON melalui \emph{Access Point} WiFi
\end{enumerate}

Pelaksanaan metodologi serta skenario pengujian yang akan dipaparkan dalam bab ini diharapkan dapat memberikan pemahaman mengenai hasil dan pembahasan sehingga dapat ditarik kesimpulan dari Tugas Akhir yang telah dilaksanakan.

\subsection{Pengujian Waktu Delay Pengiriman Data String Melalui Bluetooth}

Pengujian waktu \emph{delay} pengiriman data ini dilakukan dengan cara mengirimkan data berupa string dari laptop menuju ESP32 melalui Bluetooth. Data yang dikirimkan adalah data arah dan kecepatan yang dipisahkan dengan koma seperti yang dapat dilihat pada Persamaan \ref{eq:string-2data}.

\begin{equation}
  \label{eq:string-2data}
    Arah(char),Kecepatan(integer)
\end{equation}
