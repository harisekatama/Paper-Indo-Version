% Ubah judul dan label berikut sesuai dengan yang diinginkan.
\section{Introduction}
\label{sec:pendahuluan}

% Ubah paragraf-paragraf pada bagian ini sesuai dengan yang diinginkan.

According to the Kamus Besar Bahasa Indonesia (Indonesian Dictionary), paralysis is the weakening of the functions of body parts to the extent that they are powerless or can no longer be moved as they should \cite{Daring_2016}. Muscles along with bones, nerves, and the connecting tissues between muscles, bones, and nerves play a crucial role in controlling human body movements. If any of these tissues are disrupted, it can result in paralysis, either temporary or permanent.

There are several conditions that can lead to paralysis, such as stroke, which can cause paralysis on one side of the face, arm, and leg, Bell's Palsy, which can result in paralysis on one side of the face without affecting other body parts, brain injury that can trigger paralysis in any part of the body corresponding to the damaged part of the brain, polio that causes paralysis in the arms, legs, and respiratory muscles, and many other conditions that can lead to paralysis \cite{Pansawira_2022}.

Individuals experiencing paralysis often face challenges in their daily mobility. They require additional tools to carry out daily activities, one of which is a wheelchair. Currently, there are electric wheelchairs controlled using a joystick \cite{choi2019motion}. However, the use of a joystick may not address the issues faced by individuals with paralysis, as those with paralysis in their arms may struggle to control this type of electric wheelchair.

In facing the challenges of paralysis, it is crucial to seek solutions that can enhance the independence of those affected. One promising approach is to leverage advanced technology, such as computer vision integrated with embedded systems. By combining these two technologies, it is hoped that innovative solutions can be created, enabling individuals with paralysis to maintain independent mobility.

Computer vision is a scientific field that enables computers to "see" \cite{TIAN20201}. This technology uses cameras to identify, track, and measure targets for further image processing. Computer vision provides the ability to recognize and understand the surrounding environment. Meanwhile, embedded systems can be configured on a personal level to meet specific needs according to the problems at hand. 

The integration of this technology can be an innovative solution to the challenges at hand. In addressing these challenges, the research will focus on the development of a motor controller that can interact optimally with computer vision technology. The strategic choice of ESP32 as the main microcontroller has been made due to its ability to precisely control motor functions. Beyond its role as a motor controller, the ESP32 will also serve as a data receiver from the computer equipped with computer vision technology. Through this integration, it is expected that the motor controller can operate synergistically with the information received from the computer, creating an efficient and responsive system.
