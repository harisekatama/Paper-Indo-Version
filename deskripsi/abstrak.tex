% Mengubah keterangan `Abstract` ke bahasa indonesia.
% Hapus bagian ini untuk mengembalikan ke format awal.
\renewcommand\abstractname{Abstrak}

\begin{abstract}

  % Ubah paragraf berikut sesuai dengan abstrak dari penelitian.
  Kelumpuhan merupakan suatu keadaan dimana penderita mengalami pelemahan fungsi pada anggota tubuhnya sehingga penderita tidak bertenaga atau tidak menggerakkan anggota tubuh sebagaimana mestinya. Terdapat beberapa kondisi yang dapat mengakibatkan kelumpuhan, mulai dari penyakit seperti stroke hingga kecelakaan. Seseorang yang mengalami kelumpuhan sering kali memiliki permasalahan dalam hal mobilitas sehari-hari. Mereka memerlukan alat tambahan agar dapat beraktivitas, salah satunya adalah kursi roda. Hingga saat ini telah terdapat kursi roda elektrik yang dikendalikan dengan menggunakan \emph{joystrick}. Akan tetapi penggunaan \emph{joystick} belum dapat menjawab permasalahan dari seseorang yang mengalami kelumpuhan. Karena bagi orang yang mengalami kelumpuhan pada bagian lengan akan kekusahan dalam mengendalikan kursi roda elektrik berjenis ini. Pada penelitian ini telah dikembangkan kontroler kursi roda elektrik yang dapat digerakkan melalui teknologi visi komputer, baik dengan pose tangan maupun gestur kepala. Integrasi teknologi ini dapat menjadi solusi yang inovatif terhadap permasalahan yang sedang dihadapi. Pemilihan ESP32 sebagai mikrokontroler urama menjadi langkah strategis, karena kemampuanya dalam mengatur kerja motor dengan presisi. Selain berfungsi sebagai kontroler motor, ESP32 juga berperan sebagai perangkat penerima data dari komputer yang dilengkapi dengan teknologi visi komputer. Dari hasil pengujian, didapatkan suatu kesimpulan bahwa pengiriman dari visi komputer sebaiknya dikirim dalam bentuk JSON dan ditransmisikan menggunakan WiFi. Hal ini dilakukan karena transmisi data dalam bentuk JSON dan ditransmisikan melalui WiFi memiliki waktu delay yang terbaik, yaitu 1,032374783 detik. Melalui integrasi ini, diharapkan bahwa kontroler motor dapat beroperasi secara sinergis dengan informasi yang diterima dari komputer dan menciptakan sebuah sistem yang efisien dan responsif.

\end{abstract}

% Mengubah keterangan `Index terms` ke bahasa indonesia.
% Hapus bagian ini untuk mengembalikan ke format awal.
\renewcommand\IEEEkeywordsname{Kata kunci}

\begin{IEEEkeywords}

  % Ubah kata-kata berikut sesuai dengan kata kunci dari penelitian.
  Kursi Roda, JSON, ESP32, WiFi.

\end{IEEEkeywords}
